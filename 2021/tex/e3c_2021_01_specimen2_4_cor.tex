\textbf{\large\textsc{Exercice 4 \hfill 5 points}}

\bigskip

Soit $f$ la fonction définie sur $\R$ par:  

\[f(x)=\e^{2x} + 6 \e^{x} - 8x-4.\]

\begin{list}{\textbullet}{Dans le plan rapporté à un repère orthogonal, on considère:}
\item $\mathcal{C}_f$ la courbe représentative de la fonction $f$;
\item $\mathcal{D}$ la droite d'équation cartésienne $y=-8x-4$.
\end{list}

%\medskip

\begin{enumerate}
\item% Montrer que, pour tout $x \in \R$, $f'(x)=2\left (\e^{x}-1\right )\left (\e^{x}+4\right )$.
\begin{list}{\textbullet}{}
\item Pour tout $x \in \R$, $f'(x) = 2\e^{2x} +6\e^{x}-8$.
\item Pour tout $x \in \R$, $2\left (\e^{x}-1\right )\left (\e^{x}+4\right ) = 2\left ( \e^{2x}-\e^{x}+4\e^{x}-4\right ) = 2\e^{2x} +6\e^{x}-8$.
\end{list}

On peut donc en déduire que, pour tout $x \in \R$, $f'(x)=2\left (\e^{x}-1\right )\left (\e^{x}+4\right )$.

\item On étudier le signe de $f'(x)$ sur $\R$.

\begin{list}{\textbullet}{}
\item Pour tout x de $\R$, $\e^{x}>0$ donc $\e^{x}+4>0$.
\item On sait que $\e^{x}>1 \iff x>0$ donc $\e^{x}-1>0$ sur $]0~;~+\infty[$.
\item De plus $\e^{x}-1=0 \iff \e^{x}=1 \iff x=0$.
\end{list}

D'où le tableau de signes de $f'(x)$:

\begin{center}
\renewcommand{\arraystretch}{1.5}
\def\esp{\hspace*{2cm}}
$\begin{array}{|c | *{5}{c} |} 
\hline
x  & -\infty & \esp & 0 & \esp  & +\infty \\
\hline
f'(x) &  & \pmb{-} &  \vline\hspace{-2.7pt}{0} & \pmb{+} &    \\
\hline
\end{array}$
\renewcommand{\arraystretch}{1}
\end{center}

\item La dérivée s'annule et change de signe pour $x=0$; $f(0) = \e^{0} + 6\e^{0} -0 -4 = 3$.

On dresse le tableau de variations de la fonction $f$ sur $\R$.

\begin{center}
{\renewcommand{\arraystretch}{1.3}
\psset{nodesep=3pt,arrowsize=2pt 3}  % paramètres
\def\esp{\hspace*{1.5cm}}% pour modifier la largeur du tableau
\def\hauteur{0pt}% mettre au moins 20pt pour augmenter la hauteur
$\begin{array}{|c| *4{c} c|}
\hline
 x & -\infty & \esp & 0 & \esp & +\infty \\
 \hline
f'(x) &  &  \pmb{-} & \vline\hspace{-2.7pt}0 & \pmb{+} & \\  
\hline
  & \Rnode{max1}{~}  &  &  &  & \Rnode{max2}{~}   \\
f(x) & &  & & &  \rule{0pt}{\hauteur} \\
 &  & &   \Rnode{min}{3} & & \rule{0pt}{\hauteur}
\ncline{->}{max1}{min} \ncline{->}{min}{max2}\\
\hline
\end{array}$
}
\end{center}

\item %En déduire le signe de $f(x)$ sur $\R$.
La fonction $f$ admet pour minimum le nombre strictement positif 3, donc pour tout $x$ de $\R$, $f(x)>0$.

\item La courbe $\mathcal{C}_f$ et la droite $\mathcal{D}$ ont un point commun si et seulement si l'équation $f(x)=-8x-4$ admet une solution.

On résout cette équation:

$f(x)=-8x-4
\iff
\e^{2x} + 6 \e^{x} - 8x-4 = -8x-4
\iff
\e^{2x} + 6 \e^{x} = 0
\iff
\e^{x} \left (\e^{x}+6\right ) = 0$

Pour tout réel $x$, $\e^{x}>0$ donc le produit $\e^{x} \left (\e^{x}+6\right )$ n'est jamais nul.

L'équation n'a donc pas de solution, ce qui veut dire que la courbe $\mathcal{C}_f$ et la droite $\mathcal{D}$ n'ont pas de point commun.



\end{enumerate}


\begin{center}
\large \textbf{Figure (non demandée)}
\end{center}

\bigskip

\begin{center}
\psset{xunit=1cm,yunit=0.2cm,labelFontSize=\scriptstyle}
\def\xmin {-6}   \def\xmax {4}
\def\ymin {-5}   \def\ymax {40}
\begin{pspicture*}(\xmin,\ymin)(\xmax,\ymax)
%\psset{yMaxValue=\ymax,yMinValue=\ymin}
\psgrid[unit=1cm,subgriddiv=1,  gridlabels=0, gridcolor=lightgray](-6,-1)(4,8)
%\psaxes[linewidth=1.8pt]{->}(0,0)(1,1)[$\vec{\imath}$,d][$\vec{\jmath}$,180]
\psaxes[arrowsize=3pt 2, ticksize=-2pt 2pt,Dy=5]{->}(0,0)(-5.99,-4.99)(\xmax,\ymax) 
\uput[dl](0,0){O}
\psplot[linecolor=blue,plotpoints=5000]{\xmin}{\xmax}{2.7183 2 x mul exp 2.7183 x exp 6 mul add x 8 mul sub 4 sub}
\uput[r](1.43,27.1){\blue $\mathcal{C}_f$}
\psplot[linecolor=red,plotpoints=5000]{\xmin}{\xmax}{-8 x mul 4 sub}
\uput[dl](-1,4){\red $\mathcal{D}$}
\end{pspicture*}
\end{center}


