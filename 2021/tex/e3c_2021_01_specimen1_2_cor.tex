\textbf{\large Exercice 2 \hfill 5 points}

\bigskip

Une urne contient six jetons rouges dont un est marqué « gagnant » et quatre jetons verts dont trois d'entre eux sont marqués « gagnant ».

\begin{list}{}{On tire au hasard un jeton de l'urne et on note les évènements :}
\item $R$ : « le jeton tiré est rouge »,
\item $V$ : « le jeton tiré est vert »,
\item $G$ : « le jeton tiré est gagnant ».
\end{list}

\begin{enumerate}
\item  On modélise la situation à l'aide d'un arbre de probabilité.

\begin{center}
\bigskip
  \pstree[treemode=R,nodesepA=0pt,nodesepB=4pt,levelsep=2.5cm,treesep=1.2cm]{\TR{}}
 {
 	\pstree[nodesepA=4pt]{\TR{$R$}\ncput*{$\frac{6}{10}$}}
 	  { 
 		  \TR{$G$}\ncput*{$\frac{1}{6}$}
 		  \TR{$\overline{G}$}\ncput*{$\frac{5}{6}$}	   
 	  }
 	\pstree[nodesepA=4pt]{\TR{$V$}\ncput*{$\frac{4}{10}$}}
 	  {
 		  \TR{$G$}\ncput*{$\frac{3}{4}$}
          \TR{$\overline{G}$}\ncput*{$\frac{1}{4}$} 
     }
}
\bigskip
\end{center}

\item La probabilité de l'évènement « le jeton tiré est un jeton vert et marqué gagnant » est:

$P(V\cap G) = p(V)\times P_{V}(G)= \dfrac{4}{10}\times \dfrac{3}{4}=\dfrac{3}{10}$.

\item Soit $P(G)$ la probabilité de tirer un jeton gagnant. %Montrer que $P(G)=\dfrac{2}{5}$.

D'après la formule des probabilités totales:

$P(G) = P(R\cap G) + P(V\cap G) = \dfrac{6}{10}\times \dfrac{1}{6}+ \dfrac{3}{10} 
= \dfrac{1}{10} + \dfrac{3}{10} = \dfrac{4}{10}=\dfrac{2}{5}$.

\item Sachant que le jeton tiré est gagnant, la probabilité qu'il soit de couleur rouge est:

$P_G(R) = \dfrac{P(R\cap G)}{P(G)} = \dfrac{\frac{1}{10}}{\frac{4}{10}} = \dfrac{1}{4}$.

\item On tire maintenant, toujours au hasard et simultanément, deux jetons dans l'urne.

On cherche la probabilité que les deux jetons soient marqués « gagnant ». 

%Expliquer votre démarche.

Il y a en tout 10 jetons dont 4 gagnants.

\begin{list}{\textbullet}{}
\item Il y a $\ds\binom{10}{2}=45$ façons de tirer 2 jetons parmi 10.
\item Il y a $\ds\binom{4}{2}= 6$ façons de tirer 2 jetons gagnants parmi les 4 gagnants.
\item Le tirage se fait au hasard donc il y a équiprobabilité.

\end{list}

La probabilité que les deux jetons soient marqués « gagnant » est donc: $\dfrac{6}{45} = \dfrac{2}{15}$. 

\end{enumerate}

%\vspace{0.5cm}

