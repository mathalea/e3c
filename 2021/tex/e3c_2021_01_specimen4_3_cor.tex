\textbf{\large Exercice 3 \hfill 5 points}

\bigskip

Un magasin commercialise des canapés et des tables de salon.

Quand un client se présente, il achète au plus un canapé et au plus une table de salon. Une étude a montré que :

\begin{list}{\textbullet}{}
\item la probabilité pour qu'un client achète un canapé est $0,24$;
\item la probabilité pour qu'un client achète une table de salon quand il a acheté un canapé est $0,25$;
\item la probabilité pour qu'un client achète une table de salon quand il n'achète pas de canapé est $0,1$.
\end{list}

On choisit un client au hasard parmi ceux ayant participé à l'étude. On note :

\begin{list}{\textbullet}{}
\item $C$ l'évènement \og le client achète un canapé \fg{} et $\overline{C}$ son évènement contraire;
\item $T$ l'évènement « le client achète une table de salon \fg{} et $\overline{T}$ son évènement contraire.
\end{list}

\begin{enumerate}
\item  On construit un arbre pondéré décrivant la situation.

\begin{center}
%\bigskip 
\small
  \pstree[treemode=R,nodesepA=0pt,nodesepB=4pt,levelsep=3cm,nrot=:U]{\TR{}}
 {
 	\pstree[nodesepA=4pt]{\TR{$C$}\naput{$0,24$}}
 	  { 
 		  \TR{$T$}\naput{$0,25$}
 		  \TR{$\overline{T}$}\nbput{$\blue 1-0,25=0,75$}	   
 	  }
 	\pstree[nodesepA=4pt]{\TR{$\overline{C}$}\nbput{$\blue 1-0,24=0,76$}}
 	  {
 		  \TR{$T$}\naput{$0,10$}
 		  \TR{$\overline{T}$}\nbput{$\blue 1-0,1=0,9$}	   
     }
}
\bigskip
\end{center}

\item La probabilité que le client achète un canapé et une table de salon est:

$P(C\cap T) = P(C)\times P_C(T)=0,24\times 0,25 = 0,06$.

\item D'après la formule des probabilités totales:

$P(T) = P(C\cap T) + P(\overline C\cap T) = 0,06+0,76\times 0,1 = 0,136$


\item Dans ce magasin, le prix moyen d'un canapé est de \np{1000}~\euro{} et le prix moyen d'une table de salon est de 300~\euro{}. On note $X$ la variable aléatoire correspondant à la somme payée par le client.
	\begin{enumerate}
		\item  
\begin{list}{\textbullet}{On a quatre possibilités.}
\item Aucun achat: 		$P(X=0)= P\left(\overline C \cap \overline T\right) = 0,76\times 0,9=0,684$.
\item Achat d'une table et pas de canapé: 		$P(X=300)=P\left(\overline C \cap T\right) = 0,76\times 0,1=0,076$.
\item Achat d'un canapé et pas de table: 		$P(X=\np{1000})=P\left(C \cap \overline T\right) = 0,24\times 0,75=0,18$.
\item Achat d'un canapé et d'une table: 		$P(X=\np{1300})= P(C \cap  T) = 0,24\times 0,25= 0,06$.
\end{list}	
		
		On complète le tableau suivant donnant la loi de probabilité de $X$.

\begin{center}
\renewcommand{\arraystretch}{1.5}
\begin{tabularx}{0.8\linewidth}{|c|*{4}{>{\centering \arraybackslash}X|}}
\hline
$x_i$ 		& 0 & 300 	& \np{1000} & \np{1300}\\\hline
$P\left(X=x_i\right)$ 	& $\blue 0,684$	& $\blue 0,076$ 		& $\blue 0,18$ & $\blue 0,06$  \\ \hline
\end{tabularx}
\end{center}
		\item L'espérance de $X$ est:
		
$E(X)=\ds\sum (x_i\times p_i) = 0\times 0,684 + 300\times 0,076 + \np{1000}\times 0,18 + \np{1300}\times 0,06 =280,8$.

%Donner une interprétation de ce nombre dans le contexte de l'exercice.
Donc la dépense moyenne d'un client entrant dans le magasin est de $280,80$ \euro.
	\end{enumerate}
\end{enumerate}

\vspace{0.5cm}

