\textbf{\large\textsc{Exercice 3 \hfill 5 points}}

\bigskip

\begin{enumerate}
\item On lance deux dés cubiques équilibrés \og classiques \fg{} et on note les numéros apparaissant sur la face supérieure de chaque dé.

Soit $X$ la variable aléatoire égale au produit des numéros apparaissant sur les deux faces.

Le jeu est gagné si le produit des numéros apparaissant sur les faces supérieures des deux
dés lancés est strictement inférieur à 10.

\medskip

\begin{enumerate}
\item Donner les valeurs prises par la variable aléatoire $X$.
\item Déterminer la loi de probabilité de  $X$.
\item Déterminer la probabilité de gagner.
\end{enumerate}

\item On lance à présent deux dés spéciaux : ce sont des dés cubiques parfaitement équilibrés dont les faces sont numérotées différemment des dés classiques.

\begin{list}{\textbullet}{}
\item Les faces du premier dé sont numérotées avec les chiffres : 1, 2, 2, 3, 3, 4.
\item Les faces du deuxième dé sont numérotées avec les chiffres : 1, 3, 4, 5, 6, 8.
\end{list}

On note $Y$ la variable aléatoire égale au produit des numéros apparaissant sur les deux faces après lancer de ces deux dés spéciaux.

Déterminer $P(Y < 10)$.

\item Est-il préférable de jouer au jeu de la question 1 avec des dés classiques ou avec des dés
spéciaux ?
\end{enumerate}

\vspace{0.5cm}

