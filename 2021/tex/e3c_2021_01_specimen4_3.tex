\textbf{\large Exercice 3 \hfill 5 points}

\bigskip

Un magasin commercialise des canapés et des tables de salon.

Quand un client se présente, il achète au plus un canapé et au plus une table de salon. Une étude a montré que :

\begin{list}{\textbullet}{}
\item la probabilité pour qu'un client achète un canapé est $0,24$;;
\item la probabilité pour qu'un client achète une table de salon quand il a acheté un canapé est $0,25$;
\item la probabilité pour qu'un client achète une table de salon quand il n'achète pas de canapé est $0,1$.
\end{list}

On choisit un client au hasard parmi ceux ayant participé à l'étude. On note :

\begin{list}{\textbullet}{}
\item $C$ l'évènement \og le client achète un canapé \fg{} et $\overline{C}$ son évènement contraire;
\item $T$ l'évènement « le client achète une table de salon \fg{} et $\overline{T}$ son évènement contraire.
\end{list}

\begin{enumerate}
\item  Construire un arbre pondéré décrivant la situation.
\item Calculer la probabilité que le client achète un canapé et une table de salon.
\item Montrer que la probabilité $P(T)$ est égale à $0,136$.
\item Dans ce magasin, le prix moyen d'un canapé est de \np{1000}~\euro{} et le prix moyen d'une table de salon est de 300~\euro{}. On note $X$ la variable aléatoire correspondant à la somme payée par le client.
	\begin{enumerate}
		\item  Recopier et compléter le tableau suivant donnant la loi de probabilité de $X$.

\begin{center}
\renewcommand{\arraystretch}{1.5}
\begin{tabularx}{0.8\linewidth}{|c|*{4}{>{\centering \arraybackslash}X|}}
\hline
$x_i$ 		& 0 & 300 	& \np{1000} & \np{1300}\\\hline
$P\left(X=x_i\right)$ 	& 	& 		& 			& \\ \hline
\end{tabularx}
\end{center}
		\item Calculer l'espérance de $X$.

Donner une interprétation de ce nombre dans le contexte de l'exercice.
	\end{enumerate}
\end{enumerate}

\vspace{0.5cm}

