
\medskip

Une entreprise qui fabrique des aiguilles dispose de deux sites de production, le site A et le site B.
Le site A produit les trois-quarts des aiguilles, le site B l'autre quart. 

Certaines aiguilles peuvent présenter un défaut. Une étude de contrôle de qualité a révélé que :

\setlength\parindent{9mm}
\begin{itemize}
\item[$\bullet~~$] 2\,\% des aiguilles du site A sont défectueuses ;
\item[$\bullet~~$] 4\,\% des aiguilles du site B sont défectueuses.
\end{itemize}
\setlength\parindent{0mm}
\medskip

Les aiguilles provenant des deux sites sont mélangées et vendues ensemble par lots. 

On choisit une aiguille au hasard dans un lot et on considère les évènements suivants:

\setlength\parindent{9mm}
\begin{itemize}
\item[$\bullet~~$]$A$ : l'aiguille provient du site A ; 
\item[$\bullet~~$]$B$ : l'aiguille provient du site B ;
\item[$\bullet~~$]$D$ : l'aiguille présente un défaut.
\end{itemize}
\setlength\parindent{0mm}

L'évènement contraire de $D$ est noté $\overline{D}$.

\medskip

\begin{enumerate}
\item% D'après les données de l'énoncé, donner la valeur de la probabilité de l'évènement $A$ que l'on notera $P(A)$.
Le site A produit les trois-quarts des aiguilles donc $P(A)=0,75$.

\item On complète  l'arbre de probabilités ci-dessous:

\begin{center}
\pstree[treemode=R,nodesepB=3pt,levelsep=3.5cm]{\TR{}}
{\pstree[nodesepA=3pt]{\TR{$A$}\naput{$0,75$}}
	{\TR{$D$}\naput{$0,02$}
	\TR{$\overline{D}$}\nbput{$0,98$}
	}
\pstree[nodesepA=3pt]{\TR{$\overline{A}$}\nbput{$0,25$}}
	{\TR{$D$}\naput{$0,04$}
	\TR{$\overline{D}$}\nbput{$0,96$}
	}
}
\end{center}

\item La probabilité que l'aiguille ait un défaut et provienne du site A est:\\
$P(A\cap D) = 0,75\times 0,02 = 0,015$.

\item D'après la formule des probabilités totales:\\
 $P(D) = P(A \cap D) + P\left (\overline{A}\cap D\right ) = 0,75\times 0,02 + 0,25\times 0,04 =0,025$.
 
\item Après inspection, l'aiguille choisie se révèle défectueuse. 

La probabilité qu'elle ait été produite sur le site A est:\\
$P_D(A)=\dfrac{P(A\cap D)}{P(D)} = \dfrac{0,015}{0,025} = 0,6$.


\end{enumerate}


