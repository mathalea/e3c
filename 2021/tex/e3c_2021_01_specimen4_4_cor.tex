\textbf{\large Exercice 4 \hfill 5 points}

\bigskip

Le plan est rapporté à un repère orthonormé \Oij{} d'unité 1 cm.

On considère la droite $\mathcal D$ d'équation $x+3y-5=0$.

\medskip

\begin{enumerate}
\item Soit A le point de coordonnées $(2~;~1)$.% appartient à la droite $\mathcal D$ et tracer la droite $\mathcal D$ dans le repère \Oij.

$x_{\text A} +3y_{\text A}-5 = 3\times 2 + 3\times 1 - 5 = 0$ donc A appartient à la droite $\mathcal D$.

On détermine un 2\ieme{} point pour tracer la droite $\mathcal{D}$: si $y=0$ alors $x-5=0$ donc $x=5$. La droite passe par le point de coordonnées $(5~;~0)$.

On trace la droite dans le repère.

\item Soit $\mathcal D'$ la droite passant par le point B\,$(4~;~2)$ et perpendiculaire à la droite $\mathcal D$.

\begin{list}{\textbullet}{}
\item
Une droite d'équation $ax+by+c=0$ a pour vecteur directeur le vecteur de coordonnées $(-b~;~a)$ donc le vecteur $\vect v$ de coordonnées $(-3~;~1)$ est un vecteur directeur de la droite $\mathcal{D}$.
\item
Les droites $\mathcal{D}$ et $\mathcal{D}'$ sont perpendiculaires, donc un vecteur directeur de l'une est un vecteur normal de l'autre; donc le vecteur $\vect v\,(-3~;~1)$ est un vecteur normal à la droite $\mathcal{D}'$.
\item
La droite $\mathcal{D}'$ est donc la droite passant par le point B et de vecteur normal $\vect v$; c'est donc l'ensemble des points M\,$(x~;~y)$ tels que $\vectt{BM}$ soit orthogonal à $\vect v$.
\item
Le vecteur $\vectt{BM}$ a pour coordonnées $(x-4~;~y-2)$.

$\vectt{BM} \perp \vect{v} \iff \vectt{BM} \cdot \vect{v}=0
\iff (x-4)(-3)+(y-2)(1)=0
\iff -3x+12 +y - 2=0\\
\phantom{\vectt{BM} \perp \vect{v}}
\iff -3x+y+10=0
\iff 3x-y-10=0$
\end{list}

La droite $\mathcal{D}'$ a pour équation $3x-y-10=0$.

\item Soit H le projeté orthogonal de B sur la droite $\mathcal D$.

Les droites $\mathcal{D}$ et $\mathcal{D}'$ sont perpendiculaires donc le point H est le point d'intersection de ces deux droites; les coordonnées de H sont donc solutions du système:
$\left \lbrace
\begin{array}{l !{=} l}
x+3y-5 & 0\\
3x-y-10 & 0
\end{array}
\right .$

%Déterminer, par le calcul, les coordonnées de H.

On résout ce système:

$\left \lbrace
\begin{array}{l !{=} l}
x+3y-5 & 0\\
3x-y-10 & 0
\end{array}
\right .
\iff
\left \lbrace
\begin{array}{r !{=} l}
x & - 3y+5 \\
3(-3y+5)-y-10 & 0
\end{array}
\right .
\iff
\left \lbrace
\begin{array}{r !{=} l}
x & - 3y+5 \\
-10y+5 & 0
\end{array}
\right .\\
\phantom{\left \lbrace
\begin{array}{l !{=} l}
x+3y-5 & 0\\
3x-y-10 & 0
\end{array}
\right .}
\iff
\left \lbrace
\begin{array}{r !{=} l}
x & - 3\times 0,5+5 \\
y & 0,5
\end{array}
\right .
\iff
\left \lbrace
\begin{array}{r !{=} l}
x & 3,5 \\
y & 0,5
\end{array}
\right .$

Le point H a pour coordonnées $\left (3,5~;~0,5\right )$.

\item On considère le cercle $\mathcal C$ de diamètre [AB] et on note $\Omega$ son centre.
	\begin{enumerate}
		\item %Déterminer une équation de $\mathcal C$ ; préciser son rayon et les coordonnées de $\Omega$.
\begin{list}{\textbullet}{}
\item Le cercle $\mathcal{C}$ de diamètre [AB] a pour centre le point $\Omega$, milieu du segment [AB]; les coordonnées de $\Omega$ sont $\left (\dfrac{x_{\text A} + x_{\text B} }{2}~;~\dfrac{y_{\text A} + y_{\text B} }{2}\right ) = (3~;~1,5)$.
\item Le cercle $\mathcal{C}$ de diamètre [AB] a pour rayon

$\dfrac{\text{AB}}{2} = \dfrac{\ds\sqrt{(x_{\text B} - x_{\text A})^2 + (y_{\text B} - y_{\text A})^2}}{2}
= \dfrac{\ds\sqrt{(4-2)^2 + (2-1)^2}}{2}
= \dfrac{\sqrt{5}}{2}$
\end{list}		
		
Le cercle $\mathcal{C}$ a donc pour équation
$(x-3)^2 + (y-1,5)^2 = \left ( \dfrac{\sqrt{5}}{2}\right )^2$ c'est-à-dire\\
$(x-3)^2 + (y-1,5)^2 = 1,25$.
			
		\item D'après les questions précédentes, le triangle ABH est rectangle en H donc il est inscrit dans le cercle de diamètre [AB], appelé $\mathcal{C}$; donc le point H appartient  au cercle $\mathcal C$.
	\end{enumerate}
	
\vspace{1cm}	
	
\begin{center}
\psset{unit=1cm,labelFontSize=\scriptstyle}
\def\xmin {-2}   \def\xmax {6}
\def\ymin {-2}   \def\ymax {6}
\begin{pspicture*}(\xmin,\ymin)(\xmax,\ymax)
%\psset{yMaxValue=\ymax,yMinValue=\ymin}
\psgrid[subgriddiv=1,  gridlabels=0, gridcolor=lightgray]
\psaxes[linewidth=1.8pt]{->}(0,0)(1,1)[$\vec{\imath}$,d][$\vec{\jmath}$,180]
\psaxes[arrowsize=3pt 2, ticksize=-2pt 2pt,labels=none](0,0)(-3.99,-1.99)(\xmax,\ymax) 
\uput[dl](0,0){O}
\psset{linecolor=blue}
\pscircle(3,1.5){1.118}
\psplot{\xmin}{\xmax}{5 x sub 3 div}
\psplot{\xmin}{\xmax}{3 x mul 10 sub}
\psdots(2,1)(4,2)(3,1.5)(3.5,0.5)
\blue
\uput[dl](2,1){A} \uput[r](4,2){B} \uput[ul](3,1.5){$\Omega$} 
\uput[-70](3.5,0.5){H}
\uput[ur](-1,2){$\mathcal{D}$} \uput[dr](3,-1){$\mathcal{D}'$} \uput[ul](2,2){$\mathcal{C}$}  
\end{pspicture*}
\end{center}
	
\end{enumerate}

