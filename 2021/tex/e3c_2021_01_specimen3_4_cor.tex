\textbf{\large Exercice 4 \hfill 5 points}

\bigskip

Bob s'est fixé un objectif : participer à un marathon qui aura lieu très bientôt dans sa ville.

Pour cela, il désire programmer sa préparation au marathon de la manière suivante :

\begin{list}{\textbullet}{}
\item lors du premier entraînement, il décide de courir 20 km ;
\item il augmente ensuite, à chaque entraînement, la distance à courir de 5\,\%.
\end{list}

On peut modéliser la distance parcourue lors de ses entraînements par une suite $(d_n)$, où, pour tout entier naturel $n$ non nul, le nombre $d_n$ désigne la distance à courir en kilomètre, lors de son $n$-ième entraînement.
On a ainsi $d_1 = 20$.

\begin{enumerate}
\item $d_2 = d_1 + d_1\times \frac{5}{100} =  20 + 20\times \frac{5}{100} = 21$, et
$d_3 = d_2 + d_2\times \frac{5}{100} =  21 + 21\times \frac{5}{100} = 22,05$.

\item %Pour tout entier naturel $n$ non nul, exprimer $d_{n+1}$ en fonction de $d_n$.
On passe de $d_n$ à $d_{n+1}$ en ajoutant 5\,\%, donc en multipliant par $1+\frac{5}{100}=1,05$. 

Donc pour tout entier naturel $n$ non nul,  $d_{n+1}= 1,05\times d_n$ 

\item La suite $(d_n)$ est donc géométrique de premier terme $d_1=20$, et de raison $q=1,05$.

 On en déduit que, pour tout entier naturel $n \geqslant 1$, $d_n= d_1\times q^{n-1} = 20\times 1,05^{n-1}$.

\item La distance, arrondie à 1 m près, que va courir Bob lors de son 10\ieme{} entraînement est\\
$d_{10} = 20\times 1,05^{9}$ soit $31,027$~km.

\item La distance à courir lors d'un marathon est de $42,195$ km. Bob estime qu'il sera prêt pour la course, s'il parvient à courir au moins 43 km lors d'un de ses entraînements.

On complète le script suivant, écrit en langage Python, dont la valeur de $n$, après exécution de ce script, est le nombre minimal d'entraînements permettant à Bob d'être prêt pour le marathon.

\begin{center}
\fbox{\tt
\begin{tabular}{l}
n = 1\\
d = 20\\
while {\blue d < 43} :\\
\quad n = \blue n + 1\\
\quad d = 1.05*d
\end{tabular}
}
\end{center}

\end{enumerate}



