
\medskip

La population d'une ville A augmente chaque année de 2\,\%. 
La ville A avait \np{4600} habitants en 2010.
La population d'une ville B augmente de 110 habitants par année.
La ville B avait \np{5100} habitants en 2010.

Pour tout entier $n$, on note $u_n$ le nombre d'habitants de la ville A et $v_n$ le nombre d'habitants de la ville B à la fin de l'année $2010 + n$.

\medskip

\begin{enumerate}
\item% Calculer le nombre d'habitants de la ville A et le nombre d'habitants de la ville B à la fin de l'année 2011
\begin{list}{\textbullet}{À la fin de 2011:}
\item Le nombre d'habitants de la ville A est: \\[5pt]
$u_1=u_0 + u_0\times \dfrac{2}{100}= \np{4600} + \np{4600}\times \dfrac{2}{100}=  \np{4692}$.

\item Le nombre d'habitants de la ville B est: $v_1=v_0+110 = \np{5100}+110 = \np{5210}$.
\end{list}

\item% Quelle est la nature des suites $\left(u_n\right)$ et $\left(v_n\right)$ ?
\begin{list}{\textbullet}{}
\item Ajouter $2\,\%$, c'est multiplier par $1+\dfrac{2}{100}=1,02$; donc la suite $(u_n)$ est géométrique de raison $q=1,02$ et de premier terme $u_0=\np{4600}$.
\item On passe de $v_n$ à $v_{n+1}$ en ajoutant $110$, donc la suite $(v_n)$ est arithmétique de raison $r=110$ et de premier terme $v_0=\np{5100}$.
\end{list}

\item %Donner l'expression de $u_n$ en fonction de $n$, pour tout entier naturel $n$ et calculer le nombre d'habitants de la ville A en 2020.
La suite $(u_n)$ est géométrique de raison $q=1,02$ et de premier terme $u_0=\np{4600}$ donc, pour tout entier naturel $n$, on a: $u_n=u_0\times q^n = \np{4600}\times 1,02^n$.

Le nombre d'habitants dans la ville A en 2020 est: $u_{10}=\np{4600}\times 1,02^{10}\approx \np{5607}$.

\item %Donner l'expression de $v_n$ en fonction de $n$, pour tout entier naturel $n$ et calculer le nombre d'habitants de la ville B en 2020.
La suite $(v_n)$ est arithmétique de raison $r=110$ et de premier terme $v_0=\np{5100}$ donc pour tout entier naturel $n$, on a:
$u_n=u_0 + n\times r = \np{5100} + 110n$.

Le nombre d'habitants dans la ville B en 2020 est: $v_{10}= \np{5100}+110\times 10 = \np{6200}$.

\item On complète l'algorithme ci-dessous qui permet de déterminer au bout de combien d'années la population de la ville A dépasse celle de la ville B.

\begin{center}
\renewcommand{\arraystretch}{1.2}
\fbox{
\begin{tabular}{l}
def année () : \hspace*{3cm}\\
\quad $u =\np{4600}$\\
\quad $v =\np{5100}$\\
\quad $n = 0$\\
\quad while $\blue u \leqslant v$\\
\qquad $u = \blue u \times 1,02$\\
\qquad $u = \blue v + 110$\\
\qquad $n = \blue n+1$\\
\quad return $n$\\ 
\end{tabular}
}
\end{center}
\end{enumerate}

\bigskip

