\textbf{\large Exercice 2 \hfill 5 points}

\bigskip

Une urne contient six jetons rouges dont un est marqué \og gagnant » et quatre jetons verts dont trois d'entre eux sont marqués \og gagnant ».

\begin{list}{}{On tire au hasard un jeton de l'urne et on note les évènements :}
\item $R$ : \og le jeton tiré est rouge »,
\item $V$ : \og le jeton tiré est vert »,
\item $G$ : \og le jeton tiré est gagnant ».
\end{list}

\medskip

\begin{enumerate}
\item  Modéliser la situation à l'aide d'un arbre de probabilité.

\item Calculer la probabilité de l'évènement \og le jeton tiré est un jeton vert et marqué gagnant f».

\item Soit $P(G)$ la probabilité de tirer un jeton gagnant. Montrer que $P(G) = \dfrac{2}{5}$.

\item Sachant que le jeton tiré est gagnant, calculer la probabilité qu'il soit de couleur rouge.

\item On tire maintenant, toujours au hasard et simultanément, deux jetons dans l'urne.

Calculer la probabilité que les deux jetons soient marqués \og gagnant ». 

Expliquer votre démarche.
\end{enumerate}

\vspace{0.5cm}

