
\medskip

Soit $h$ la fonction définie sur [0~;~26] par
$h(x) = - x^3 + 30x^2 - 108x - 490.$

\smallskip

\begin{enumerate}
\item %Soit $h'$ la fonction dérivée de $h$.
$h'(x) = -3 x^2 +30 \times 2x - 108 = -3x^2 +60 x - 108$

\item  On note $\mathcal{C}$ la courbe représentative  de $h$ et $\mathcal{C}'$ celle de $h'$.
	\begin{enumerate}
		\item %Identifier $\mathcal{C}$ et $\mathcal{C}'$ sur le graphique orthogonal ci-dessous parmi les trois courbes $\mathcal{C}_1$,  $\mathcal{C}_2$ et $\mathcal{C}_3$ proposées.
La courbe $\mathcal C$ représentant la fonction $h$ est la courbe $\mathcal{C}_2$.		
		
La courbe $\mathcal{C}'$ représentant la fonction $h'$ est la courbe $\mathcal{C}_1$.		
		
		\item% Justifier le choix  pour $\mathcal{C}'$.
$h(0)=-490$ donc la courbe $\mathcal{C}_2$ représente la fonction $h$; on peut donc voir que la fonction $h$ est décroissante, puis croissante, puis décroissante. La fonction dérivée $h'$ sera donc négative, puis positive, puis négative. Elle est donc représentée par la courbe $\mathcal{C}_1$.		
		
		
	\end{enumerate}
		
\begin{center}
\psset{xunit=0.3cm,yunit=0.003cm}
\begin{pspicture*}(-8,-600)(28,1500)
\psaxes[linewidth=1.25pt,labelFontSize=\scriptstyle,Dx=2,Dy=200]{->}(0,0)(-7.99,-599)(28,1500)
\psplot[plotpoints=2000,linewidth=1.25pt,linecolor=red]{-8}{28}{x dup mul 30 mul x 3 exp sub 108 x mul sub 490 sub}
\psplot[plotpoints=2000,linewidth=1.25pt,linecolor=blue]{-8}{28}{60 x mul x dup mul 3 mul sub 108   sub}
\psplot[plotpoints=2000,linewidth=1.25pt,linestyle=dashed]{-8}{28}{60 x mul x dup mul 3 mul  sub 108   sub neg}
\uput[ul](12,806){$\red \mathcal{C}_2$} \uput[ur](14,144){$\blue \mathcal{C}_1$} 
\uput[d](12,-200){$\mathcal{C}_3$} 
\end{pspicture*}
\end{center}

\item Soit $(T)$ la tangente à $\mathcal{C}$ au  point A d'abscisse $0$. %Déterminer son équation réduite.

La droite $(T)$ a pour équation réduite: $y= h'(0)\left (x-0\right ) + h(0)$.

$h(0)= -490$ et $h'(0) = -108$

La droite $(T)$ a donc pour équation réduite: $y= -108x - 490$.

\item %Étudier le signe de $h'(x)$ puis dresser le tableau de variation de la fonction $h$ sur [0~;~26].
$h'(x) = -3x^2+60x-108$ est un polynôme de degré 2 dont le discriminant est\\
$\Delta = b^2-4ac = 60^2 - 4\times (-3)\times (-108) = \np{2304}=48^2$

Ce polynôme admet donc deux racines:\\
$x'=\dfrac{-b+\sqrt{\Delta}}{2a} = \dfrac{-60+48}{-6}=2$ et 
$x''=\dfrac{-b-\sqrt{\Delta}}{2a} = \dfrac{-60-48}{-6}=18$

On en déduit le signe de $h'(x)$ puis le sens de variation de $h$.

$h(0)=-490$, $h(2)=-594$, $h(18)=1454$ et $h(26)=-594$

\begin{center}
{\renewcommand{\arraystretch}{1.3}
\psset{nodesep=3pt,arrowsize=2pt 3}  % paramètres
\def\esp{\hspace*{1.5cm}}% pour modifier la largeur du tableau
\def\hauteur{20pt}% mettre au moins 20pt pour augmenter la hauteur
$\begin{array}{|c| *6{c} c|}
\hline
 x & 0 & \esp & 2 & \esp & 18 & \esp & 26 \\
 \hline
h'(x) &  &  \pmb{-} & \vline\hspace{-2.7pt}0 & \pmb{+} & \vline\hspace{-2.7pt}0 & \pmb{-} & \\  
\hline
  &   \Rnode{max1}{-490} & &  & & \Rnode{max2}{\np{1454}} & & \\
h(x) & &  & & & & & \rule{0pt}{\hauteur} \\
 & & & \Rnode{min1}{-594} & & & & \Rnode{min2}{-594}  \rule{0pt}{\hauteur}
\ncline{->}{max1}{min1} 
\ncline{->}{min1}{max2}
\ncline{->}{max2}{min2} \\
\hline
\end{array}$
}
\end{center}

\end{enumerate}



\bigskip

