\textbf{\large Exercice 4 \hfill 5 points}

\bigskip

Dans un repère orthonormé \Oij{}, on considère les points A\,$(3~;~1)$, B\,$(-3~;~3)$ et C\,$(2~;~4)$.

\begin{enumerate}
\item Montrer que l'équation $x + 3y - 6 = 0$ est une équation cartésienne de la droite (AB).
\item Déterminer une équation cartésienne de la droite $d$, perpendiculaire à la droite (AB) et
passant par le point C.
\item En déduire les coordonnées du point $K$, projeté orthogonal du point C sur la droite (AB).
\item Calculer la distance AB et déterminer les coordonnées du milieu M du segment [AB].
\item En déduire une équation du cercle de diamètre [AB].
\end{enumerate}
