\textbf{\large\textsc{Exercice 2 \hfill 5 points}}

\bigskip

Une collectivité locale octroie une subvention de \np{116610}~\euro{} pour le forage d'une nappe d'eau souterraine. Une entreprise estime que le forage du premier mètre coûte 130~\euro; le
forage du deuxième mètre coûte 52~\euro{} de plus que celui du premier mètre ; le forage du
troisième mètre coûte 52~\euro{} de plus que celui du deuxième mètre, etc.

Plus généralement, le forage de chaque mètre supplémentaire coûte 52~\euro{} de plus que celui
du mètre précédent.

Pour tout entier $n$ supérieur ou égal à 1, on note: $u_n$ le coût du forage du $n$-ième mètre en
euros et $S_n$ le coût du forage de $n$ mètres en euros; ainsi $u_1=130$.

\medskip

\begin{enumerate}
\item Calculer $u_2$ et $u_3$.
\item Préciser la nature de la suite $\left(u_n\right)$. En déduire l'expression de $u_n$ en fonction de $n$, pour tout $n$ entier naturel non nul.
\item Calculer $S_2$ puis $S_3$.
\item Afin de déterminer le nombre maximal de mètres que l'entreprise peut forer avec la subvention qui est octroyée, on considère la fonction Python suivante:

\begin{center}
\fbox{\tt
\begin{tabular}{l}
def nombre\_metre(S) :\\
\quad C = 130\\
\quad n = 1\\
\quad while C < S :\\
\qquad C = C + ...\\
\qquad n = n + 1\\
 \quad return n
\end{tabular}
}
\end{center}
Compléter cet algorithme de sorte que l'exécution de la fonction \verb!nombre_metre(S)! renvoie le nombre maximal de mètres que l'entreprise peut forer avec la subvention octroyée. Justifier votre réponse.

\item On admet que, pour tout entier naturel non nul, $S_n=26n^2 +104n$. En déduire la valeur de
$n$ que fournit la fonction Python donnée à la question 4. On expliquera la démarche utilisée.
\end{enumerate}


