\textbf{\large\textsc{Exercice 4 \hfill 5 points}}

\bigskip

Soit $f$ la fonction définie sur $\R$ par: 

\[f(x)=\e^{2x} + 6 \e^{x} - 8x-4.\]

\begin{list}{\textbullet}{Dans le plan rapporté à un repère orthogonal, on considère:}
\item $\mathcal{C}_f$ la courbe représentative de la fonction $f$;
\item $\mathcal{D}$ la droite d'équation cartésienne $y=-8x-4$.
\end{list}

\medskip

\begin{enumerate}
\item Montrer que, pour tout $x \in \R$, $f'(x)=2\left (\e^{x}-1\right )\left (\e^{x}+4\right )$.
\item Étudier le signe de $f'(x)$ sur $\R$.
\item Dresser le tableau de variations de la fonction $f$ sur $\R$.
\item En déduire le signe de $f(x)$ sur $\R$.
\item La courbe $\mathcal{C}_f$ et la droite $\mathcal{D}$ ont-elles un point commun ? Justifier.
\end{enumerate}

